	\section{Реализация программного решения}

		\subsection{Пример работы реализации}

	Данный пример иллюстрирует работу группы лифтов, где их количество равно 2, в здании с 5 этажами.
		В ходе работы системы появится два человека в моменты времени $t_2$ и $t_4$ на этажах $e_1$ и $e_0$,
		и у каждого человека целью будет четвёртый этаж $d_4$.

	Изначально первая кабина находится на первом этаже $e_0$, а вторя на втором $e_1$.
		Ниже приведён лог показывающий работу системы:

	Изучив выше изложенный журнал, можно увидеть результат, что каждый человек доставлен и ожидание составило не более одной единицы времени.


		\subsection{Реализация на SimPy}

			Данная симуляция системы управления группой лифтов с ограниченным количеством кабин и несколькими людьми,
				которые приходят для попадания с одного этажа на другой.
				В данной системе используется ресурс для моделирования ограниченного количества кабин.
				Он также определяет процесс выбора кабины и перевозку ей человека.

			Когда человек появляется рядом с шахтой лифта, он вызывает первую из свободных кабин.
				Как только он его кабина забирает, время ожидания человека прекращается.
				Он, наконец, добирается до нужного этажа и уходит.

			После старта системы начинается генерация людей, они появляются после случайного интервала времени,
			пока продолжается симуляция.


			При том, что данная программа является многопоточной, стоит отметить тот факт, что здесь используются общие ресурсы.
				Эти ресурсы могут использоваться для ограничения количества процессов, использующих их одновременно.
				Процесс должен запрашивать право использования ресурса. Как только право на использование больше не требуется,
				оно должно быть выпущено. Данная система смоделирована как ресурс с ограниченным количеством кабин.
				Люди прибывают к кнопке вызова кабины и просят выполнить перевозку на другой этаж.
				Если все кабины заняты, человек должен ждать, пока один из пассажиров не закончит поездку и не освободит кабину.

				\subsubsection{Пример работы реализации}

			Данный пример иллюстрирует работу группы лифтов, где их количество равно 2, в здании с 30 этажами.
				В ходе работы системы появится люди в случайные моменты времени на случайных этажах этажах,
				и у каждого человека целью является добраться на другой этаж.

			Изначально первая кабина находится на первом этаже $e_0$, а вторя на втором $e_1$.
				Ниже приведён лог показывающий работу системы:

			Изучив выше изложенный журнал, можно увидеть, что за время симуляции появилось 4 человека,
				каждый человек доставлен.

		\subsection{Интеллектуальная реализация}

		Однако, пусть реализация логического вывода на языке Prolog является целесообразной задачей, но для разделения моделируемой системы на логический блок и блок взаимодействия объектов необходима клиент-сервергая связка. А реализация сервера или клиента на языке Prolog не является его типовой задачей, что и касается реализации графической составляющей системы моделирования.

			Таким образом более целесообразном будет оставить блок взаимодействия объектов реализованными на языку Prolog.
				Более того, правила описанные для модуляции системы будут использованы
				для построения решения логическим блоком.

			Данный код иллюстрирует реализацию формулы реализующие обход возможных вариантов будущего, сбор статистики с каждого варианта и выбор наиболее подходящего варианта будущего по признаку. в данном случае интересующим признаком является среднее ожидание человеком кабины. 

% \input{src/pl_mod_code.tex}

	Благодаря функционалу SWI-Prolog
		предикаты отражающие состояние системы в момент вызова кабины можно будет указывать в правилах только при необходимости.

		\subsubsection{Пример работы реализации}

	Данный пример иллюстрирует работу группы лифтов, где их количество равно 2, в здании с 5 этажами.
		В ходе работы системы появится два человека в моменты времени $t_2$ и $t_4$ на этажах $e_1$ и $e_0$,
		и у каждого человека целью будет четвёртый этаж $d_4$.

	Изначально первая кабина находится на первом этаже $e_0$, а вторя на втором $e_1$.
		Ниже приведён лог показывающий работу системы:

		%example

Изучив выше изложенный журнал, можно увидеть, что происходит построение дерева формулы и её обход.
	Для того чтобы было нагляднее следует прокомментировать строку лога.

Первые четыре столбца в логе - это реальное время журналирования момента модуляции.
	Следующим столбцом идёт связка двух чисел, первое число - это номер процесса,
	он необходим для идентификации сессии, а второе число показывает момент времени модуляции.
	Ещё одним столбцом является связка строки и числа, число - это так же момент времени в данной ветки формулы,
	А строка отражает индекс чанка формулы в момент вывода, r означает корень выводимой формулы, а дальше серез нижние подчёркивание перечислены индексы кабин, которые участвуют в логическом выводе в данные момент.

