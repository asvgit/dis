	\section{Используемые технологии}

	В качестве реализации языка Prolog был выбран SWI-Prolog.
		Это открытая реализация, работающая на Unix, MacOS и Windows.
		В данной реализации имеется необходимы функционал, а именно предикаты nb\_getval, nb\_setval и nb\_current.
		Эти предикаты позволяют использовать другие предикаты в качестве глобальных переменных, что
		позволяет существенно упростить написание правил для вычисления.

	В процессе реализации было создано порядка десяти файлов с описанием правил логического вывода
		общей суммой порядка семиста строк. Такое большое количество правил обусловлено наличием деталей и нюансов.
		Ниже представлен основной файл, в котором реализован инициализация логического вывода.
		Данный код иллюстрирует реализацию формулы времени. Благодаря функционалу SWI-Prolog
		в дальнейшем предикаты отражающие время можно будет указывать в правилах только при необходимости.

	Реализации правил manage\_people и manage\_elevators вынесены в другие файлы,
		как как являются весьма громоздкими.
		Manage\_people отвечает за внешний фактор (случайное появление нуждающихся в лифте людей).
		А manage\_elevators включает в себя формулы движения лифтов.
\subsection{SimPy}

SimPy - это Python фреймворк процессо-ориентированной дискретно-событийной системы моделирования. Его диспетчеры событий основаны на функциях-генераторах Python.

Также они могут использоваться для создания асинхронных сетей или для реализации мультиагентных систем (с как моделируемым, так и реальным взаимодействием).

Процессы в SimPy - это просто Python генераторы, которые используются для моделирования активных компонентов, например, таких как покупатели, транспортные средства или агенты. SymPy также обеспечивает различные виды общих ресурсов для моделирования точек с ограниченной пропускной способностью (например, серверов, касс, тоннелей). Начиная с версии 3.1, SimPy также будет обеспечить возможности мониторинга для помощи в сборе статистических данных о ресурсах и процессах. 

SymPy представляет собой открытую библиотеку символьных вычислений на языке Python. Цель SymPy - стать полнофункциональной системой компьютерной алгебры (CAS), при этом сохраняя код максимально понятным и легко расширяемым. SymPy полностью написан на Python и не требует сторонних библиотек.

SymPy можно использовать не только как модуль, но и как отдельную программу. Программа удобна для экспериментов или для обучения. Она использует стандартный терминал IPython, но с уже включенными в нее важными модулями SymPy и определенными переменными x, y, z.
