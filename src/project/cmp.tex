\section{Сравнительный анализ}

	Как было упомянуто выше одним из основных критериев является средняя длительность ожиданий.
		Так же можно сравнить время выполнения модуляции. С помощью среднего времени ожидания
		можно оценить эффективность разработанного решения, и, используя время выполнения программ можно
		произвести оценку ресурсозатратности разработанного решения.

	Проведения анализа необходимо провести тестирование, имеющихся программных решений.
		По результатам данного тестирования можно будет произвести оценку эффективности и ресурсозатратности.

	\subsection{Условия проведения тестирования}

		Основными исходными данными являются: количество кабин, количество людей участвовавших при моделировании,
			точнее количество поступивших запросов, и количество  обслуживаемых этажей.
			Все упомянутые данные можно менять и производить большое количество разноплановых тестов. 

		В данном тестировании в качестве констант были взяты количество лифтов и количество людей.
			С помощью такого рода тестов можно произвести оценку не только упомянутых свойств системы, но и 
			увидеть какова типа инфраструктура нуждается или не нуждается в разработанных решениях.

		Так же был выбран ряд значений для количества обслуживаемых этажей: 30, 15 и 5.
			Для зданий со сложной инфраструктурой подходит значение в 30 этажей, для зданий с менее развитой
			инфраструктурой 15 этажей и для ещё менее 5 этажей.

		 Для сбора дынных производился многочисленный запуск модуляции и был произведён подсчёт средних значений.
		 	  Результаты сбора данных приведены в таблицах \ref{projectt1}-\ref{projectt4}.
			  Для упрощения представления полученных данных были даны названия реализациям: более интеллектуальным способом
			  реализации является I тип, а более тривиальным является II тип.

	\subsection{Сравнение реализаций}

		Не взирая на тот факт, что модели типов I и II сильно упрощены, уже можно произвести оценку их
			эффективности и ресурсозатратности  по данным в таблицах \ref{projectt1}  и \ref{projectt2}.
		
		{
\changefontsizes[12pt]{12pt}
\captionsetup{font=large,margin=21pt}

\vspace{14pt}
\begin{longtable}[t]{@{\extracolsep{\fill}}|l|@{\hskip+35pt}p{0.15\textwidth}|@{\hskip+35pt}p{0.15\textwidth}|@{\hskip+35pt}p{0.15\textwidth}|}
% \begin{longtable}[t]{@{\extracolsep{\fill}}|l|@{\hskip-28pt}c|@{\hskip-28pt}c|@{\hskip-28pt}c|}
	\caption{Сравнение по среднему времени ожидания \vspace{-35pt}} \label{projectt1} \\ \hline
			&&&\\[-7pt]
	Способ управления
		& 30 этажей \hspace{14pt}
			& 15 этажей \hspace{14pt}
				& 5 этажей  \hspace{14pt}  \\  \hline
	\endfirsthead
	\caption* {Продолжение таблицы \ref{projectt1}\vspace{-35pt}}\\ \hline
			&&&\\[-7pt]
	Способ управления
		& 30 этажей
			& 15 этажей
			& 5 этажей   \\ \hline \endhead 
			&&&\\[-7pt]
	I тип     &	7.5		&	6	& 2.7	\\ \hline
			&&&\\[-7pt]
	II тип    &	9.6		&	7.1	& 3.2		\\ \hline
\end{longtable}
}


			По результатам из таблицы \ref{projectt1}  можно сделать вывод, что более интеллектуальная реализация
				оказалась эффективнее тривиальной во всех условиях.

		{
\changefontsizes[12pt]{12pt}
\captionsetup{font=large,margin=10mm}

\vspace{14pt}
\begin{longtable}[t]{@{\extracolsep{\fill}}|l|@{\hskip+35pt}p{0.15\textwidth}|@{\hskip+35pt}p{0.15\textwidth}|@{\hskip+35pt}p{0.15\textwidth}|}
	\caption{Сравнение по времени выполнения  \vspace{-35pt}} \label{projectt2} \\ \hline
			&&&\\[-7pt]
	Способ управления
		& 30 этажей \hspace{14pt}
			& 15 этажей \hspace{14pt}
				& 5 этажей  \hspace{14pt}  \\  \hline
	\endfirsthead
	\caption* {Продолжение таблицы \ref{projectt2}\vspace{-35pt}}\\ \hline
			&&&\\[-7pt]
	Способ управления
		& 30 этажей
			& 15 этажей
			& 5 этажей   \\ \hline \endhead 
			&&&\\[-7pt]
	I тип     &	20.516 с	&	3,711 с	& 0,527 с	\\ \hline
			&&&\\[-7pt]
	II тип    &	0,153 с		&	0,134 с	& 0,112 с	\\ \hline
\end{longtable}
}

			
			Однако, по результатам из таблицы \ref{projectt2}  очевидно, что более интеллектуальная реализация
				оказалась в десятки раз медленнее, чем более простая реализация.

	\subsection{Сравнение реализаций с элементом вероятности}
		
		Для тестирования реализаций с элементом вероятности появления запроса, было добавлено 5 человек (запросов),
			имеют низкую вероятность появления.

		При таких условиях возникает вопросы:
		\begin{itemize}
			\item[--] будет ли система  I типа всё ещё эффективной;
			\item[--] на сколько сильно процесс проведения модуляции затянется, если затянется.
		\end{itemize}

		Конечно, было бы нечестно реализации I типа не дать возможность оценки вероятности появления запроса.
			Поэтому было добавлено правило учитывать вероятность появления запроса.

		{
\changefontsizes[12pt]{12pt}
\captionsetup{font=large,margin=21pt}

\vspace{14pt}
\begin{longtable}[t]{@{\extracolsep{\fill}}|l|@{\hskip+35pt}p{0.15\textwidth}|@{\hskip+35pt}p{0.15\textwidth}|@{\hskip+35pt}p{0.15\textwidth}|}
% \begin{longtable}[t]{@{\extracolsep{\fill}}|l|@{\hskip-28pt}c|@{\hskip-28pt}c|@{\hskip-28pt}c|}
	\caption{Сравнение по среднему времени ожидания \vspace{-35pt}} \label{projectt3} \\ \hline
			&&&\\[-7pt]
	Способ управления
		& 30 этажей \hspace{14pt}
			& 15 этажей \hspace{14pt}
				& 5 этажей  \hspace{14pt}  \\  \hline
	\endfirsthead
	\caption* {Продолжение таблицы \ref{projectt3}\vspace{-35pt}}\\ \hline
			&&&\\[-7pt]
	Способ управления
		& 30 этажей
			& 15 этажей
			& 5 этажей   \\ \hline \endhead 
			&&&\\[-7pt]
	I тип     &	7.8		&	5.5	& 3.4	\\ \hline
			&&&\\[-7pt]
	II тип    &	9.6		&	7.1	& 3.2	\\ \hline
\end{longtable}
}


		По результатам из таблицы \ref{projectt3}  можно сделать вывод, что более интеллектуальная реализация
			остаётся всё ещё эффективнее тривиальной, но  при небольшом числе этажей интеллектуальная реализация
			оказалась менее эффективной.

		{
\changefontsizes[12pt]{12pt}
\captionsetup{font=large,margin=21pt}

\vspace{14pt}
\begin{longtable}[t]{@{\extracolsep{\fill}}|l|@{\hskip+35pt}p{0.15\textwidth}|@{\hskip+35pt}p{0.15\textwidth}|@{\hskip+35pt}p{0.15\textwidth}|}
% \begin{longtable}[t]{@{\extracolsep{\fill}}|l|@{\hskip-28pt}c|@{\hskip-28pt}c|@{\hskip-28pt}c|}
	\caption{Сравнение по времени выполнения  \vspace{-35pt}} \label{projectt4} \\ \hline
			&&&\\[-7pt]
	Способ управления
		& 30 этажей \hspace{14pt}
			& 15 этажей \hspace{14pt}
				& 5 этажей  \hspace{14pt}  \\  \hline
	\endfirsthead
	\caption* {Продолжение таблицы \ref{projectt3}\vspace{-35pt}}\\ \hline
			&&&\\[-7pt]
	Способ управления
		& 30 этажей
			& 15 этажей
				& 5 этажей   \\ \hline \endhead 
			&&&\\[-7pt]
	I тип     &	56.791  с	&	40.739 c	& 32.957 c	\\ \hline
			&&&\\[-7pt]
	II тип    &	0.171 c		&	0.149 c		& 0.127 c		\\ \hline
\end{longtable}
}


		Более того, по результатам из таблицы \ref{projectt4}  видно, что более интеллектуальная реализация
			оказалась ещё медленнее, в то время как тривиальная реализация в показателях не изменилась.


		Полученные результаты можно оправдать тем, что интеллектуальная реализация тратила время на обработку
			запросов, которые по факту не произошли, и это не помешало её превзойти более тривиальную реализацию
			на больших данных.

	\subsection{Сравнение реализаций на языках Prolog и Python}

		Входе выполнения экспериментов на выполнение модуляций с ещё большими числами не хватило вычислительной мощи
			оборудования для реализаций на Prolog. Что иллюстрирует потребность в ресурсах у данных подходов
			к решению задачи.
		
		Стоит отметить, что программное решение было выполнено и на языке Python, которое ни обладало
			особой интеллектуальностью и не не несло в себе какого-то нового алгоритма. Поведённые тесты
			показали следующий результат: среднее время ожидания -- 12.2 и время выполнения -- 0.065 секунд.

		Хоть и результат на языке Python, и является более быстрым, но в то же время имеет наихудший результат
			по среднему времени ожидания. Наверняка, попытка вынести функции интеллектуальной системы 
			на более быструю платформу будет целесообразным.
