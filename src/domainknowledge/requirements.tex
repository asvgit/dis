\section{Требования к системе управления лифтами}

	Основными требованиями, предъявляемым к системе управления лифтовым оборудованием является безопасность работы,
		надёжность, плавность разгона, движения и торможения,
		точность остановки кабины, минимизация шума во время работы и недопущение помех
		радиоприему и телевидению. Эти требование необходимо учитывать,
		как при проектировании системы управления, так и в процессе монтажа и эксплуатации.
		
	Ключевое требование безопасность работы лифта. При его работе есть вероятность возникновения некоторых аварийных ситуаций, таких как превышение скорости перемещения кабины выше допустимой, произвольный пуск лифта, перегрузка кабины, пуск лифта при открытых дверях кабины и/или шахты, обрыв каната либо элементов подвеса. Таким образом, алгоритм, по которому работает система управления, должен предусматривать возникновение подобных ситуаций.
	
	Наилучшим алгоритмом работы является тот, который обеспечивает общедоступность пользования и комфортабельность пассажиров. Общедоступность пользования лифтом предполагает наличие достаточно простой и понятной системы управления движением из кабины и этажных площадок, не требующей специальной подготовки пассажиров всех возрастных групп. Комфортабельность условий перевозки определяется минимальным значением времени ожидания кабины пассажирами на посадочном этаже и перемещения их между этажами.

	При этом алгоритм работы системы должен минимизировать расход электроэнергии и при этом не требовать значительных финансовых затрат.

	Наиболее распространенный алгоритм должен предполагать:
		\begin{itemize}
			\item[--] исключение направления на этаж вызова более одной кабины;
			\item[--] выполнение вызова на определенный этаж назначается идущей в нужном направлении не полностью загруженной кабине. Если таковой нет, ближайшей свободной кабине;
			\item[--] автоматическое направление первой из освободившихся кабин на этаж наибольшего спроса (обычно - первый), а остальные кабины после освобождения остаются на этажах, на которых они пришли по приказам.
		\end{itemize}
	
	Также необходимо учитывать, что режим работы главного электропривода лифта характеризуется частым включением и отключением. При этом выделяются такие этапы движения:
		\begin{itemize}
			\item[--] разгон электродвигателя до установившейся скорости;
			\item[--] движение с установившейся скоростью;
			\item[--] уменьшение скорости при подходе к этажу назначения;
			\item[--] торможение и остановка кабины лифта на указанном этаже с требуемой точностью.
		\end{itemize}
	
	Наряду с этим имеет место тот факт, что этап движения с установившейся скоростью может отсутствовать, если сумма путей разгона до установившейся скорости и торможения меньше расстояния между этажами отправления и назначения.

	Исходя из этого можно заключить, что в зависимости от условий работы целесообразно проектировать систему, в которой реализованы различные скорости движения.
	
			Так, например, в зависимости от назначения рекомендуется применять пассажирские лифты со следующими номинальными скоростями:
		\begin{itemize}
			\item[--] в административных зданиях и гостиницах: до 9 этажей - от 0,7 м/с до 1 м/с; от 9 до 16 этажей - от 1 до 1,4 м/с;
			\item[--] в административных зданиях от 16 этажей –1,6 – 4 м/с.
		\end{itemize}
	
	Так как система, разрабатываемая в рамках данной работы, предназначена для двадцатиэтажного здания, то примем скорость лифта равно 1,6 м/с. Исходя из таблиц основных параметров и применяемости лифтов в зависимости от вида здания, которые приведены в ГОСТ 5746-2015, примем грузоподъёмность равной 1000 кг при вместимости кабины 12 человек, подразумевая, что лифтовое оборудование будет применяться в общественных, административных зданиях или в зданиях промышленных предприятий.
	
	Также к системе управления лифтовым оборудованием предъявляются некоторые требования по пожарной безопасности. В случае возникновения пожара все лифты должны опуститься на главный посадочный этаж и открыть двери (в качестве главного посадочного этажа определён первый). В таком состоянии лифты блокируются до тех пор, пока не будет отключена система пожарной безопасности.
