\section{Системы управления группой лифтов}
	На сегодняшний день существует большое количество систем управления лифтами,
		в числе которых есть как системы релейного типа, характерные для старых построек,
		так и системы, на основе микропроцессорной техники.
		Все системы управления можно разделить на два типа в зависимости от их структуры:
		централизованные и распределенные.

	\subsection{Централизованные системы группового управления лифтами}
		Система данного типа подразумевает наличие одного главного (центрального) контроллера,
			который размещается в станции управления и обеспечивает функционирование системы управления в целом.
			В некоторых случаях возможно применение небольших локальных плат или контроллеров,
			обеспечивающих какие-либо примитивные функции управления, но, в основном,
			их функции сводятся к формированию сигналов для отправки на центральный пост
			управления и локальному распределению управляющих сигналов.
			
		Примером такой системы может являться станция управления лифтами,
			разработанная компанией STEP Electric Corporation.
			Данная станция разработана на основе технологии централизованного группового управления,
			при этом групповой контроллер производит определение места расположения лифтовой площадки,
			регистрацию и обработку команд, которые могут быть получены как от индивидуальных контроллеров лифтов,
			так и напрямую от устройств, расположенных в кабинах или на этажах.

		Достоинствами централизованной системой управления является высокий уровень планирования
			и контроля функционирования системы. Однако при сложной системе могут проявиться некоторые недостатки:
		\begin{itemize}
			\item[--] возникновение, так называемого, «эффекта бутылочного горла»,
				когда обработка информации и выдача управляющего воздействия могут
				быть продолжительными по времени по причине перегрузки центрального пункта управления.
			\item[--] увеличение длительности «цикла управления» из-за отдаленности
				места принятия решений от места их исполнения.
			\item[--] большое количество проводов, поскольку все сигналы необходимо
				направить на центральный пост управления.
		\end{itemize}
		
		По состоянию на сегодняшний день в мировой лифтовой индустрии наблюдается отказ
		от централизованных систем управления и перехода к децентрализованным (распределенным). 

	\subsection{Распределенные системы управления лифтами}

	Значительное снижение цен на микропроцессоры и полупроводниковые микросхемы и приборы,
		и столь же существенное увеличение производительности микропроцессоров,
		наблюдаемые на протяжении нескольких лет, сделали актуальными и рентабельными
		распределенные системы и системы реального времени на базе микроконтроллеров.

	Недостатки централизованного управления могут быть скомпенсированы применением
		распределенной системы, которая характеризуется наличием ряда иерархически,
		функционально, структурно связанных контроллеров.

	В такой системе управление децентрализовано и добавление нового контроллера
		в сеть системы не ограничено при условии ее грамотной организации.
		При этом общая задача управления разбивается на несколько подзадач,
		выполнение каждой из которых производится отдельный контроллером. Решение этих задач выполняется одновременно.

	При этом интеграция новых устройств как собственного производства,
		так и других производителей становится более реализуемой,
		а аппаратные решения более эффективными – добавление дополнительного
		функционала решается использованием дополнительных пунктов контроля и управления (контроллеров).

	В качестве примера данной структуры может служить система управления грузопассажирским лифтом,
		которая была разработана компанией Элеси.

	Система выполнена как распределенная на современной элементной базе,
		как следствие экономия кабельной продукции и повышение надежности функционирования.

	Ключевыми достоинствами систем данного типа является сокращение
		числа проводов, возможность оперативно адаптации системы к частным условиям и требованиям
		за счет соответствующего распределения программного обеспечения,
		повышенная помехоустойчивость системы программным способом без применения специальных средств.

	Недостатками этой системы может являться то,
		что групповое управление лифтами может сводится лишь к оповещению контроллеров
		других лифтов о выполнении какой-либо операции, что объясняется отсутствием
		единого центра диспетчеризации. Другими словами, организация эффективного алгоритма
		управления в ней ставится под вопрос. 

	Еще одним примером распределенной системы управления лифтами
		могут быть системы управления ЛиРа. Данная система имеет в своем составе
		центральный шкаф управления для выполнения диспетчеризации и множество плат,
		используемых под каждую конкретную функцию, среди которых
		есть плата телефонной связи, плата тормоза, плата ключей и так далее.

	Достоинства и недостатки данной системы вытекают из ее строения.
		С одной стороны, широкий выбор специализированных плат даёт
		возможность собрать систему любой необходимой конфигурации.
		С другой, большое количество плат может сделать систему сложной
		и неудобной для монтажа и последующего обслуживания.
