\section{Методы управления группой лифтов}

	В зависимости от принятой системы управления лифты могут работать индивидуально, в паре или в группе. 

	При индивидуальной работе действие одного лифта не влияет на действие другого.
		Для индивидуально работающих лифтовых установок характерны свойства:
		\begin{itemize}
			\item[--] системы вызовов на таких установках действуют только на один лифт;
			\item[--] одиночное управление, как правило, применяется для обслуживания зданий с небольшим числом этажей.
		\end{itemize}

	При парной или групповой работе лифты действуют зависимо друг от друга.
		Такая система предназначена для тех случаев, когда одиночная установка лифта недостаточна
		для обеспечения пассажиропотока и в здании устанавливают два лифта, обслуживающие одни и те же этажи.
		Системы парного управления обеспечивают такую согласованную работу двух лифтов с общей системой вызовов,
		при которой достигается максимальная производительность их и минимальное время ожидания.

	Групповая система управления применяется при групповой установке лифтов в крупных
		общественных и административных зданиях, когда имеют место напряженные
		пассажиропотоки и парные установки не обеспечивают требуемой производительности.
		Групповая установка применяется с числом лифтов от трех до четырех.

	В общем виде программа работы одного или группы лифтов сводится к отработке кабинами отдельных команд:
		вызовов, приказов и др. Программы работ в разных режимах для наиболее массовых
		типовых пассажирских лифтов жилых зданий приводятся в техническом описании принципа
		действия их электрических схем.

	Система вызовов на групповых лифтах так же, как и на парных,
		общая для всей группы совместно работающих лифтов, то есть устанавливается только одна
		вызывная кнопка на этаже. Кроме того, групповые лифты оборудованы специальным автоматическим
		устройством организации совместной работы лифтов. 

	В системах группового управления предусматриваются утренний,
		дневной и вечерний режимы работы. Эти режимы задаются диспетчером
		или устанавливаются автоматически в зависимости от направленности и напряженности пассажиропотока в здании.
