\begin{longtable}[t]{@{\extracolsep{\fill}}|l|l|l|l|}
	\caption{Воздействие климатических факторов} \label{taskt2} \\ \hline
	{\No} & \shortstack{Наименование\\ воздействующего\\ фактора \vspace{27pt}}
			  & \shortstack{Характеристика\\ воздействующего\\ фактора \vspace{25pt}}
				& \shortstack{\\ Максимальное\\ значение (диапазрн\\ возможных измерений)\\ воздействующего\\ фактора}   \\ \hline
	\endfirsthead
	\caption* {Продолжение таблицы \ref{taskt2}}\\ \hline
	{\No} & \shortstack{Наименование\\ воздействующего\\ фактора \vspace{27pt}}
			  & \shortstack{Характеристика\\ воздействующего\\ фактора \vspace{25pt}}
				& \shortstack{\\ Максимальное\\ значение (диапазрн\\ возможных измерений)\\ воздействующего\\ фактора}   \\ \hline \endhead
	1     & \shortstack{\\ Синусоидальная\\ вибрация} & \shortstack{\\ диапазон частот, Гц}  & $0,5 - 200 * 10^8$     \\ \hline
	2     & \shortstack{\\ Случайная\\ вибрация} & \shortstack{\\ диапазон частот, Гц}       &  $0,5 - 200 * 10^{20}$    \\ \hline
	3     & \shortstack{\\ Удары\\ многократного\\ действия} & \shortstack{\\ максимальная амплитуда\\ ускорения, $m*c^2, (g)$ } & 10     \\ \hline
	4     & \shortstack{\\ Удары\\ одиночного\\ действия} & \shortstack{\\ максимальная амплитуда\\ ускорения, $m*c^2, (g)$ } & 20     \\ \hline
	5     & \shortstack{\\ Линейное\\ ускорение} & \shortstack{\\ максимальная амплитуда\\ ускорения, $m*c^2, (g)$ } & 30     \\ \hline
\end{longtable}
