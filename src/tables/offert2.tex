\begin{longtable}[t]{@{\extracolsep{\fill}}|l|@{\hskip-14pt}p{0.3\textwidth}|@{\hskip-14pt}p{0.3\textwidth}|}
	\caption{Сравнение видов датчиков} \label{offert2} \\ \hline
	Вид & Преимущества & Недостатки \\ \hline
	\endfirsthead
	\caption* {Продолжение таблицы \ref{offert2}}\\ \hline
	Вид & Преимущества & Недостатки \\ \hline
	\endhead
	Инфракрасные	&
		Возможность довольно точной регулировки дальности и угла обнаружения движущихся объектов

		Удобен в использовании вне помещений т.к. реагирует лишь на объекты имеющие собственную температуру

		При работе абсолютно безопасны для здоровья человека или домашних питомцев, т.к. работает как «приемник», ничего не излучая
					&
		Возможность ложных срабатываний. Из-за того, что датчик реагирует на любые ИК (тепловые) излучения, могут случаться ложные срабатывания даже на теплый воздух, поступающий из кондиционера, радиаторов отопления и т.п.

		Снижена точность работы на улице. Из-за воздействия окружающих факторов, таких как прямой солнечный свет, осадки и т.п.

		Относительно небольшой диапазон рабочих температур

		Не обнаруживает объекты облаченные/покрытые не пропускающими ИК - излучение материалами
		\\ \hline
	Ультразвуковые	&
		Относительно невысокая стоимость

		Не подвергаются влиянию окружающей среды

		Определяют движение вне зависимости от материала объекта

		Имеют высокую работоспособность в условиях высокой влажности или запылённости

		Не зависят от влияния температуры окружающей среды или объектов
					&
		Многие домашние животные слышат ультразвуковые частоты, на которых работает датчик движения, что зачастую вызывает у них сильный дискомфорт

		Относительно невысокая дальность действия

		Срабатывает только на достаточно резкие перемещения, если двигаться совсем плавно – возможно обмануть ультразвуковой датчик движения
		\\ \hline
	Микроволновые	&
		Имеет более высокую стоимость относительно датчиков других типов с аналогичными показателями

		Возможность ложных срабатываний, из-за движений вне необходимой зоны наблюдения, за окном и т.п.

		СВЧ излучение небезопасно для здоровья человека
					&
		Датчик способен обнаруживать объекты за разнообразными диэлектрическими или слабо проводящими ток препятствиями: тонкими стенами, дверьми, стеклами и т.п.

		Работоспособность датчика не зависит от температуры окружающей среды или объектов

		Микроволновый датчик движения способен реагировать на самые незначительные движения объекта

		Датчик обладает более компактными размерами

		Может иметь несколько независимых зон обнаружения
		\\ \hline
\end{longtable}
