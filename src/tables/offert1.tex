\begin{longtable}[t]{@{\extracolsep{\fill}}|l|@{\hskip-14pt}p{0.3\textwidth}|@{\hskip-14pt}p{0.3\textwidth}|}
\changefontsizes[14pt]{14pt}
	\caption{Сравнение реализаций} \label{offert1} \\ \hline
	Вид & На ПЛИС & На микроконтроллере и компораторе \\ \hline
	\endfirsthead
	\caption* {Продолжение таблицы \ref{offert1}}\\ \hline
	Вид & На ПЛИС & На микроконтроллере и компораторе \\ \hline
	\endhead
	Структура	&
		1) ПЛИС хранит установленные настройки и суммирует значения датчиков. 

		2) Сравнение установленных настроек и показателей датчиков производится посредством ПЛИС. 

		3) Результат сравнения преобразуется из цифрового сигнала в аналоговый.
					&
		1) Цифровой микроконтроллер хранит установленные настройки. 

		2) Значения с датчиков проходят через сумматор. 

		3) Сравнение установленных настроек, хранимых в микроконтроллере, и показателей датчиков производится с помощью компаратора. 

		4) Выходная информация с компаратора преобразуется в аналоговый сигнал.
		\\ \hline
	Преимущества	&
		Низкое энергопотребление
					&
		Простой ремонт, низкая стоимость
		\\ \hline
	\shortstack{\\ Примерная\\ стоимость}	&
		1750
					&
		1200
		\\ \hline
\end{longtable}
