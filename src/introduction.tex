\chapter*{Введение}\addcontentsline{toc}{chapter}{Введение}
	На сегодняшний день неотъемлемой частью технического оснащения практически любого здания
		является подъемное устройство. Это может быть эскалатор, подъемная платформа,
		но самым распространенным оборудованием является лифт.

	Лифт -- это грузоподъемное устройство, предназначенное для вертикального перемещения
		грузов и людей в кабине, передвигающейся по жестким направляющим.
		По конструктивным особенностям лифты бывают различными, но несмотря на это действуют они по одному принципу.

	В устройстве любого лифта обязательно присутствуют определенные компоненты,
		такие как кабина (или платформа), она закрепляется на стальных тросах,
		перекинутых через шкив приводного механизма, передающего силу с одного места на другое.
		В машинном отделении в верхней части шахты лифта расположен сам приводной механизм
		вместе с аппаратурой управления лифтом, куда и передаются сигналы из кабины лифта. 

	Лифты подразделяются на категории по видам транспортируемых грузов.

	Существуют лифты пассажирские, используемые в жилых, общественных зданиях.
		В пассажирском лифте допускается перевозка легких грузов и предметов домашнего обихода при условии,
		что их общая масса вместе с пассажиром не превышает грузоподъемности лифта.
		Такие лифты могут быть также с увеличенной платформой - для перевозки больных и инвалидов,
		поскольку они обязательно должны вмещать на платформу инвалидную коляску или кушетку.

	Еще одной категорией являются грузовые лифты, которые также могут подразделяться
		по видам прилагаемой подъемной силы (к верхней части кабины, либо к нижней).

	Лифты могут также классифицироваться по способу обслуживания.
		В этом случае различают лифты самостоятельного пользования,
		которыми управляет сам пассажир, и лифты, которыми управляет проводник и которые всегда сопровождают груз.

	Разные лифты движутся с различной скоростью. По скорости движения кабины они подразделяются на следующие типы:
		\begin{itemize}
			\changefontsizes[14pt]{14pt}
			\item[--] тихо­ходные (до 1,0 м/с);
			\item[--] быстроходные (от 1,0 до 2,0 м/с);
			\item[--] скоростные (от 2,0 до 4,0 м/с);
			\item[--] высокоскоростные (свыше 4,0 м/с).
			\vspace{-14pt}
		\end{itemize}

	Относительно типа привода подъемного механизма лифты подразделяются на электрические
		(с приводом от электродвигателя пе­ременного или постоянного тока)
		и гидравлические (с приводом в виде подъемного гидроцилиндра или лебедки с гидродвига­телем вращательного типа).

	В устройстве лифта важна не только механика и принцип движения кабины в пространстве,
		но также и способы реагирования лифта на поступающие от пользователей запросы.
		По способу управления лифты различают:
		\begin{itemize}
			\changefontsizes[28pt]{14pt}
			\item[--] простое раздельное управление. В данном случае регистрируется
				и выполняется только одна команда (приказ или вызов);
			\item[--] собирательное управление, при котором регистрируются все команды,
				а их выполнение осуществляется в соответствии с программой работы лифта.
				При этом могут совершаться попутные остановки по вызовам или приказам.
				Для лифтов жилых зданий попутные остановки по вызовам выполняются только
				при движении кабины вниз, а в общественных зданиях - в обоих направлениях.
				По приказам попутные остановки предусмотрены во всех лифтах в обоих направлениях;
			\item[--] одиночное управление (управление одним лифтом);
			\item[--] групповое - управление группой лифтов, расположенных в одной шахте,
				обслуживающих одни и те же этажи и имеющих одинаковую скорость.
				Разновидностью группового управления является парное управление лифтами,
				применяемое в жилых зданиях повышенной этажности.
			\vspace{-14pt}
		\end{itemize}

	В данной работе будут рассмотрены проблемы группового управления лифтами.

	Целью выпускной квалификационной работы является разработка программного решения,
		подтверждающего принципиальную возможность и целесообразность построения
		и использования системы логического вывода в интеллектуальном управлении группой лифтов.

	В рамках данной цели необходимо выполнить следующие задачи:
		\begin{itemize}
			\changefontsizes[14pt]{14pt}
			\item[--] изучить принципы автоматического логического вывода;
			\item[--] разработать математическую модель интеллектуального управления группой лифтов;
			\item[--] разработать варианты программной реализации модели интеллектуального управления группой лифтов;
			\item[--] провести сравнительный анализ полученный программных решений.
			\vspace{-14pt}
		\end{itemize}
