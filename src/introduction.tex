\chapter*{Задание}\addcontentsline{toc}{chapter}{Задание}
Выполнение курсовой работы направлено на приобретение навыков по разработке технической документации, сопровождающей создаваемый продукт. 
Техническая документация к любому продукту в определенных аспектах основывается на существующих стандартах и нормативных документах, регулирующих то или иное направление функционирования продукта. 
Данные нормативные документы необходимы для обеспечения разработки легального, безопасного, экономически обоснованного и эффективного программного обеспечения. Задача студента составить полное описание продукта и его работы, включающее стадию его разработки и стадию эксплуатации. Курсовая работа должна содержать разделы, освещающие назначение продукта, технические требования к нему и окружению, обеспечивающие надёжность и безопасность работы, конструктивные требования и проектные решения, требования к эксплуатации и условия технического обслуживания, а также технико-экономическое обоснование продукта. 

Для выполнения курсовой работы студент должен ознакомиться с нормативной базой ГОСТов, СНИП и СП, выявить необходимые документы для регулирования функционирования продукта и прописать технические требования и условия для разработки данного продукта.

\chapter*{Введение}\addcontentsline{toc}{chapter}{Введение}
Для внедрения разрабатываемого продукта необходим перечень документов.
Данные документы необходимы для разработки, внедрения и сопровождения разрабатываемого продукта.
В данной курсовой работе разрабатывается перечень документов для охранной системы, а именно:
\begin{enumerate}
	\item Техническое задание;
	\item Техническое предложение;
	\item Технический проект;
	\item Схема функциональная;
	\item Схема структурная;
	\item Схема принципиальная;
	\item Чертёж основания;
	\item Чертёж крышки корпуса;
	\item Чертёж сборочный;
	\item Рисунок печатной платы.
\end{enumerate}

Общеизвестным фактом является то, что с автоматизированными системами рука об руку идёт и программное сопровождение.
Следовательно, в стадии разработки нового продукта следует включить и этапы разработки программного обеспечения.
А также следует разработать требования к программной части продукта и включить в проект программную реализацию.

Для того, чтобы разрабатываемый продукт обладал конкурентоспособностью, уделяют большое внимание на его качество и соответствие принятым стандартам.
Преследуя выше сказанную цель, разрабатываемый продукт подвергают сертификации.
Для успешного завершения процесса разработки и внедрения нового продукта следует ориентироваться на перечень нормативных документов, например:
\begin{itemize}
	\item ГОСТ 34.602 89\cite{gost89};
	\item ГОСТ 34.601.90\cite{gost90}:
	\item ГОСТ 34.603.92\cite{gost92};
	\item ГОСТ 2.114-95\cite{gost95};
	\item ГОСТ Р 53736-2009\cite{gost09}.
\end{itemize}

Сейчас существует разнообразное множество стандартов и рекомендаций, которые находятся в свободном доступе.
В процессе выполнения данной курсовой работы следует ознакомиться с некоторым перечнем нормативных документов и получить 
навыки в разработки продуктов с программным обеспечением.
