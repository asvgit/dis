\chapter{Техническое задание}
	\section{Обоснование для проведения работ}
		Задание преподавателя.
	\section{Исполнитель работ}
		Студент ИрГУПС ФУТиИT группы ПИь-16 Арляпов С.В.
	\section{Цель выполнения работ}
		Разработать устройство охранной сигнализации, предназначенное для контроля, мониторинга и управления территориально-распределенными объектами муниципальных и ведомственных образований с целью увеличения безопасности.
	\section{Назначение продукции}
		Устройство будет обеспечивать сбор, обработку, передачу и представление в заданном виде служебной информации и информации о проникновении (попытки проникновения).
	\section{Технические требования}
		\subsection{Состав продукции}
		 	\begin{enumerate}
				\item Основной блок устройства обработки информации;
				\item Датчики «Рапид 3» - 3 штуки;
				\item Витая пара – 1 км;
				\item Вилки «RJ-45» - 5 штук;
				\item Эксплуатационная документация.
			\end{enumerate}
		\subsection{Требования к показателям назначения}
			\subsubsection{Выполняемые функции}
				Разрабатываемое устройство должно обеспечивать в режиме реального времени:
				\begin{enumerate}
					\item сбор данных;
					\item обработку данных;
					\item оповещение оператора.
				\end{enumerate}
			\subsubsection{Нормы и количественные показатели}
				\begin{itemize}
					\item Время реакции не менее 1 секунды.
					\item Время срабатывания механизма оповещения не менее 30 секунд.
					\item Дальность обнаружения датчиком не менее 15 метров.
				\end{itemize}
			\subsubsection{Технические характеристики (параметры)}
				Максимальное количество подключаемых датчиков не менее 4 штук.

				Максимальная длина кабеля, подключающего датчик к устройству, 400 метров
			\subsubsection{Требования к совместимости}
				Особых требований не предъявляется.

			\subsubsection{Требования по мобильности}
				Разрабатываемое изделие должно быть выполнено в стационарном исполнении.
		\subsection{Требования к электропитанию}
			Электропитание осуществляется по первой категории надежности от однофазной (трехфазной) сети переменного тока 220В, 50Гц, от отдельной группы электрощита, находящегося в охраняемом помещении.
		\subsection{Требования надежности}
			\subsubsection{Требования по безотказности}
				Разрабатываемое устройство должно удовлетворять следующим требованиям:
				\begin{itemize}
					\item вероятность безотказной работы 0,95, не менее;
					\item средняя наработка на отказ 50000 часов, не менее;
					\item среднее время восстановления 1 час, не более.
				\end{itemize}
			\subsubsection{Требования по долговечности}
				Разрабатываемое устройство должно удовлетворять следующим требованиям: срок службы до списания 8 лет, не менее.
			\subsubsection{Критерии отказов и предельного состояния изделия}
				Отказом разрабатываемого изделия считают невыполнение функций, заданных требованиями п.5.2.1 настоящего технического задания.
		\subsection{Конструктивные требования}
			\subsubsection{Конструктивное исполнение входящих в разрабатываемое устройство должно обеспечивать:}
				\begin{enumerate}
					\item Удобство эксплуатации;
					\item Возможность ремонта.
				\end{enumerate}
				\subsubsection{Разрабатываемое изделие должно иметь моноблочную конструкцию.}
				\subsubsection{Разрабатываемое изделие должно соответствовать следующим требованиям:}
					5.5.3.1 Размеры:
					\begin{enumerate}
						\item Габаритные - 450х450х160 мм, не более;
						\item Установочные - 500х500х210 мм, не более.
					\end{enumerate}
					5.5.3.2 Масса - 2 кг, не более.
					5.5.3.3 Устройство крепится на вертикальную поверхность с помощью винтовых соединений.
					5.5.3.4 Тип кабеля – витая пара.
					5.5.3.5 Тип порта - RJ-45.
					5.5.3.6 Разрабатываемое изделие должно иметь максимальную длину кабеля, подключающего датчик к устройству, 400 метров.
					\subsubsection{Покрытия должны обеспечивать необходимую коррозионную стойкость, надежную работу и декоративный вид разрабатываемого изделия при эксплуатации и при хранении.}
					\subsubsection{Оборудование не должно требовать доступа сзади при монтаже, подводке кабеля и обслуживании.}
					\subsubsection{Внешние электрические разъемы должны иметь маркировку, позволяющую определить их назначение.}
					\subsubsection{Электрическая схема должна быть выполнена на единой печатной плате. Монтаж должен осуществляться с помощью методов групповой пайки.}
		\subsection{Требования по эргономике и технической эстетике}
			Кодирование и компоновка средств отображения информации, органов управления на пульте управления, цветовое оформление лицевых панелей пульта разрабатываемого изделия должны обеспечивать безошибочность и быстродействие операторов, удобство и безопасность работы в любое время суток.  Необходимо предусмотреть независимое автономное питание, обеспечивающее работу ПКП и извещателей в течении не менее чем 24 часов в дежурном режиме и в течении не менее чем 3 часов в режиме «тревога».
		\subsection{Требования к эксплуатации, удобству технического обслуживания и ремонта}
			\subsubsection{Требования к стойкости к внешним воздействующим факторам}
				5.7.1.1 Разрабатываемое изделие должно быть стойким, устойчивым и прочным к воздействию климатических факторов в соответствии с таблицей \ref{taskt1}:
				\renewcommand{\arraystretch}{1.5}
\begin{longtable}[t]{@{\extracolsep{\fill}}|l|l|l|l|}
	\caption{Воздействие климатических факторов} \label{taskt1} \\
	\hline
	{\No} & \shortstack{Наименование\\ воздействующего\\ фактора \vspace{27pt}}
			  & \shortstack{Характеристика\\ воздействующего\\ фактора \vspace{25pt}}
				& \shortstack{\\ Максимальное\\ значение (диапазрн\\ возможных измерений)\\ воздействующего\\ фактора}   \\ \hline
	\multicolumn{4}{|c|}{Стойкость}    \\ \hline
	1     & \shortstack{\\ Температура\\ окружающей среды} & Градусов цельсия                                                             & 5 до 35     \\ \hline
	2     & Влажность воздуха                           & \shortstack{\\ Относительная\\ влажность при\\ температуре $25^{\circ}$, \%}             & до 70       \\ \hline
	3     & Атмосферное давление                        & Па (мм рт. ст.)                                                              & 630 до 800  \\ \hline
	\multicolumn{4}{|c|}{Устойчивость}   \\ \hline
	4     & \shortstack{\\ Температура\\ окружающей среды} & Градусов цельсия                                                             & 40          \\ \hline
	6     & Влажность воздуха                           & \shortstack{\\ Относительная\\ влажность при\\ температуре $25^{\circ}$, \%}             & до 80       \\ \hline
	9     & Атмосферное давление                        & Па (мм рт. ст.)                                                              & 800 до 900  \\ \hline
	\multicolumn{4}{|c|}{Прочность}   \\ \hline
	7     & \shortstack{\\ Температура\\ окружающей среды} & Градусов цельсия                                                             & 50          \\ \hline
	8     & Влажность воздуха                           & \shortstack{\\ Относительная\\ влажность при\\ температуре $25^{\circ}$, \%}             & до 90       \\ \hline
	9     & Атмосферное давление                        & Па (мм рт. ст.)                                                              & 900 до1000  \\ \hline
\end{longtable}


				5.7.1.2 Разрабатываемое изделие должно быть устойчивым к воздействию механических факторов в соответствии с таблицей \ref{taskt2}:
				\begin{longtable}[t]{@{\extracolsep{\fill}}|l|l|l|l|}
	\caption{Воздействие климатических факторов} \label{taskt2} \\ \hline
	{\No} & \shortstack{Наименование\\ воздействующего\\ фактора \vspace{27pt}}
			  & \shortstack{Характеристика\\ воздействующего\\ фактора \vspace{25pt}}
				& \shortstack{\\ Максимальное\\ значение (диапазрн\\ возможных измерений)\\ воздействующего\\ фактора}   \\ \hline
	\endfirsthead
	\caption* {Продолжение таблицы \ref{taskt2}}\\ \hline
	{\No} & \shortstack{Наименование\\ воздействующего\\ фактора \vspace{27pt}}
			  & \shortstack{Характеристика\\ воздействующего\\ фактора \vspace{25pt}}
				& \shortstack{\\ Максимальное\\ значение (диапазрн\\ возможных измерений)\\ воздействующего\\ фактора}   \\ \hline \endhead
	1     & \shortstack{\\ Синусоидальная\\ вибрация} & \shortstack{\\ диапазон частот, Гц}  & $0,5 - 200 * 10^8$     \\ \hline
	2     & \shortstack{\\ Случайная\\ вибрация} & \shortstack{\\ диапазон частот, Гц}       &  $0,5 - 200 * 10^{20}$    \\ \hline
	3     & \shortstack{\\ Удары\\ многократного\\ действия} & \shortstack{\\ максимальная амплитуда\\ ускорения, $m*c^2, (g)$ } & 10     \\ \hline
	4     & \shortstack{\\ Удары\\ одиночного\\ действия} & \shortstack{\\ максимальная амплитуда\\ ускорения, $m*c^2, (g)$ } & 20     \\ \hline
	5     & \shortstack{\\ Линейное\\ ускорение} & \shortstack{\\ максимальная амплитуда\\ ускорения, $m*c^2, (g)$ } & 30     \\ \hline
\end{longtable}


			\subsubsection{Требования к эксплуатационным показателям}
				5.7.2.1 Разрабатываемый Комплекс должен обеспечивать циклическую работу со следующими параметрами цикла: время загрузки – 30 мин., время обработки – 10 час., время выгрузки – 30 мин., время подготовки – 10 мин.
				5.7.2.2 Должен быть обеспечен режим работы от аварийного источника питания.
				5.7.2.3 Периодическое техническое обслуживание разрабатываемого изделия должно проводиться не реже одного раза в год.
				5.7.2.4 Периодическое техническое обслуживание должно включать в себя обслуживание всех датчиков.
				5.7.2.5 К обслуживанию комплекса должны допускаться лица, имеющие допуск к работе с электроустановками напряжением до 220 В.
				5.7.2.6 Гарантийный срок разрабатываемого Комплекса должен составлять 5 лет, не менее.
			\subsubsection{Требования по ремонтопригодности}
				5.7.3.1 Обслуживание и ремонт разрабатываемого Изделия должны производиться без применения специальных инструментов."
				5.7.3.2 Требования к ЗИП
				\begin{enumerate}
					\item Комплект ЗИП должен включать запасные части, необходимые для ремонта и поддержания работоспособного состояния разрабатываемого изделия в течение одного года.
					\item В комплект ЗИП должны входить дополнительные датчики и вилки RJ-45.
				\end{enumerate}
		\subsection{Требования безопасности}
			\subsubsection{Условия работы персонала разрабатываемой Системы должны соответствовать санитарным нормам по СанПиН 2.2.2/2.4.1340-03.}
			\subsubsection{Требования безопасности при монтаже, наладке, эксплуатации, обслуживании и ремонте разрабатываемого Комплекса должны быть приведены в эксплуатационной документации.}
		\subsection{Требования к упаковке и маркировке}
			\subsubsection{Требования к упаковке}
				Упаковка должна быть выполнена из картона материалов и обеспечивать защиту от ударных воздействий.
			\subsubsection{Требования к маркировке}
				5.9.2.1 Надписи, цифры, буквы и знаки, нанесенные при маркировке, должны быть хорошо видны, и сохранять четкость в течение всего срока эксплуатации.
				5.9.2.2 Маркировка упаковки для транспортирования должна содержать основные, дополнительные, информационные надписи и манипуляционные.
		\subsection{Требования к консервации, хранению и транспортированию}
			\subsubsection{Условия хранения}
				Изделие должно храниться в упакованном виде в отапливаемых и вентилируемых помещениях при температуре от 5 до 35 °С и относительной влажности воздуха не выше 80\% (при температуре 25 °С) при отсутствии в этих помещениях конденсации влаги, паров химически активных веществ и источников электромагнитных полей.
			\subsubsection{Срок хранения}
				Срок хранения разрабатываемого изделия в условиях отапливаемых хранилищ в соответствии с паспортными данными на аппаратуру, но не менее 8 лет.
			\subsubsection{Условия транспортирования:}
				\begin{itemize}
					\item Температура окружающей среды: от минус 50 до 50 °С;
					\item Относительная влажность до 95 \% при температуре 30 °С;
					\item Атмосферное давление от 84 до 107 кПа (от 630 до 800 мм рт. ст.);
					\item Воздействие ударных нагрузок многократного действия с пиковым ускорением не более 15g (147 м/с2) при длительности действия ударного ускорения 10–15 мс.
				\end{itemize}
				\subsubsection{Гарантийный срок хранения разрабатываемого прибора в заводской упаковке в отапливаемом помещении}
					Не менее одного года.
		\subsection{Требования стандартизации, унификации и каталогизации}
			Особых требований не предъявляется.
	\section{Требования по видам обеспечения}
		\subsection{Требования по метрологическому обеспечению}
			Особых требований не предъявляется.
		\subsection{Требования по программному обеспечению}
			Особых требований не предъявляется.
	\section{Экономическое обоснование}

	\section{Наименование этапов и выполняемых работ}
		\begin{enumerate}
			\item Техническое предложение:
				\begin{enumerate}
					\item Выбор датчиков;
					\item Выбор структурной схемы;
					\item Выбор оптимального варианта реализации;
					\item Разработка и согласование с преподавателем комплекта технической документации, разрабатываемой в рамках договора;
					\item Разработка ТД в соответствии с согласованном комплектом.
				\end{enumerate}
			\item Технический проект:
				\begin{enumerate}
					\item Разработка технического проекта, в том числе:
						\begin{itemize}
							\item Разработка конструктивных решений Комплекса и его составных частей:
								\begin{itemize}
									\item Разработка чертежей;
									\item Разработка функциональной и принципиальной схемы.
									\item Создание рисунка печатной платы.
								\end{itemize}
							\item Выполнение необходимых расчетов для технических решений, обеспечивающих показатели надежности.
						\end{itemize}
					\item Разработка эксплуатационной документации в соответствии с согласованном перечнем.
				\end{enumerate}
		\end{enumerate}
