\schap{Заключение}

	Несмотря на тот факт, что более сложные решения в управлении лифтовыми группами нуждаются в большом объёме ресурсов,
		интеллектуальные системы управления позволяют своевременно справляться с пассажиропотоком в среде с быстро
		меняющимися условиями. А также подобная система позволит учитывать вновь возникшие условия и принципы
		среды, в которой происходит управление, без существенных изменений в программной реализации, что снижает
		стоимость поддержки данной системы. Плюс, упрощается процесс разработки сложных алгоритмов управления.

	Полученные при программном моделировании результаты показали, что посредством добавления дополнительных условий
		можно улучшать интересуемые показатели, например: среднее время ожидания кабины пассажиром и
		характеристики построения логического вывода.

	Принцип рассмотрения вариантов возможного будущего, позволяет сделать систему управления более гибкой в отношении к
		особенностям пассажиропотока, зависящими от времени суток или другим временным факторам.

	Разработанные программные решения могут быть использованы для моделирования более сложных сред для проведения
		экспериментов и построения гипотез.

	Следующим шагом в развитии данной работы будет разбиение разработанного программного решения на два бока: логический и
		физический. Где логический блок будет отвечать за принятие решения, а физический блок за физические процессы,
		протекаемые в построенной модели. После чего можно будет вести разработку двух блоков независимо друг от друга.
		Что приведёт к использованию новой оптимизированной интеллектуальной системы управления и замене физической
		модели на реальную систему лифтов.
